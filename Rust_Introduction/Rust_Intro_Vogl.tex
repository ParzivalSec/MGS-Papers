% !TEX encoding = IsoLatin2
\documentclass[Seminar, MGS, german]{twbook}
\usepackage[T1]{fontenc}
% Hier kann je nach Betriebssystem eine der folgenden Optionen notwendig sein, um die Umlaute korrekt wiederzugeben:
% utf8, latin, applemac
\usepackage[ansinew]{inputenc}
% Die nachfolgenden 2 Pakete stellen sonst nicht ben�tigte Features zur Verf�gung
\usepackage{blindtext}
\usepackage{abstract}

\title{Firmengr�ndung in Wien\\Der Weg zum eigenen Spielestudio}
\author{Lukas Vogl, BSc.}
\studentnumber{gs16m007}
\supervisor{Helmut Gaberschek}
\place{Wien}

\begin{document}
\maketitle

\chapter{Einleitung}

Ein Traum den sich viele Hobby-VideospielentwicklerInnen und StudentInnen teilen ist der eines eigenen Videospielstudios. Mit der Gr�ndung einer Firma und dem Schaffen eines Studios hoffen viele, ihre Ideen verwirklichen zu k�nnen und ihre ganze Zeit und Aufmerksamkeit einem eigenen Projekt zukommen zu lassen. Doch neben der Freiheit zur kreativen Entfaltung und dem Gef�hl der Kontrolle �ber die Spielidee �bernimmt man mit der Gr�ndung einer Firma eine gro�e Verantwortung. Um den Prozess der Firmengr�ndung in Wien besser zu verstehen versucht der Autor in dieser Arbeit einen Leitfaden zu erstellen
in dem der Gr�ndungsprozess beschrieben wird. Dabei bedient er sich an Unterlagen der Stadt Wien anhand derer die einzelnen Schritte des Gr�ndungsprozess umrissen, sowie verschiedene F�rderungen der Stadt Wien beschrieben werden. Anschlie�end wird die aktuelle Marktsituation in �sterreich anhand ausgew�hlter Studios betrachtet.

\chapter{Gr�ndungsprozess erl�utert}

Ist die Entscheidung zur Gr�ndung einer eigenen Firma gefallen steht man als baldige/r Gr�nderIn vor der Frage wo man beginnen soll. Um den Prozess der Firmengr�ndung verst�ndlich zu machen und die Vorteile und Nachteile diverser Entscheidungsm�glichkeiten aufzuzeigen hat der Autor sich dazu entschieden den Prozess in mehrere Unterpunkte zu unterteilen die bei Entscheidungen in dem jeweiligen Punkten helfen sollen. Bei Unklarheiten im Prozess soll es einen Referenzpunkt geben an dem man alle Informationen findet die zur Entscheidungsfindung hilfreich sind. 

\section{Gr�ndungsvoraussetzungen und Vorbereitungen}
\blindtext  
\section{Gr�ndungskosten und Kapitalbedarf}
\blindtext  
\section{Rechtliche Grundlagen}
\blindtext  
\subsection{Rechtsformen}
\blindtext  
\subsection{Steuern}
\blindtext  
\section{Gr�ndung und Beh�rdenwege}
\blindtext  
\section{Beratungsstellen und Hilfsnetzwerke}
\blindtext  

\chapter{F�rderungen und Finanzierungsm�glichkeiten}

\section{Finanzierungsm�glichkeiten}
\blindtext  
\subsection{Business Angles}
\blindtext  
\subsection{Crowdfunding}
\blindtext  
\subsection{Venture Captial}
\blindtext  
\subsection{Bank}
\blindtext  

\section{F�rderungen in Wien}
\subsection{F�rderungsm�glichkeiten}
\blindtext  

\chapter{Aktuelle Marktsituation in �sterreich}
\section{Markt�bersicht �sterreich}
\blindtext  
\section{Markt�bersicht Wien}
\blindtext  
\subsection{Sproing}
\blindtext  
\subsection{Mipumi}
\blindtext  
\subsection{Indie-Entwickler Szene}
\blindtext  

\clearpage
\bibliographystyle{plain}
\begin{thebibliography}{99}
	
WKO Gr�nderservice: https://www.gruenderservice.at/

Schein: https://publikationen.technikum-wien.at/volltexte/2014/4843/pdf/24872\_gs10m013.pdf

\end{thebibliography}
\clearpage

% Das Abbildungsverzeichnis
\listoffigures
\clearpage

% Das Tabellenverzeichnis
\listoftables
\clearpage

\phantomsection
\addcontentsline{toc}{chapter}{Abk�rzungsverzeichnis}
\chapter*{Abk�rzungsverzeichnis}
\begin{acronym}[XXXXX]
	\acro{ABC}[ABC]{Alphabet}
	\acro{WWW}[WWW]{world wide web}
	\acro{ROFL}[ROFL]{Rolling on floor laughing}
\end{acronym}
\end{document}